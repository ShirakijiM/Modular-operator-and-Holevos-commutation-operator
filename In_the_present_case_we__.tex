In the present case we have $\mathfrak{H}_2=\mathfrak{H}$, but when we consider an infinite
dimensional Hilbert space ${\cal H}$ the equality does not hold, i.e. $\mathfrak{H}_2\subset \mathfrak{B}({\cal H})\subset \mathfrak{H}$. This makes it difficult to extend the discussion in Sec. 4 to the infinite dimensional case.

As stated in Sec. 3, we can define the
modular operator $\Delta$ for the von Neumann algebra $\mathfrak{M}=\ell(\mathfrak{N})$ and its cyclic separating vector $\rho^{1/2} \in \mathfrak{H}_2$. 
In this paper we derive a simple relation between such derived modular operator and the commutation operator:
$$
    \Delta=\left(1+\frac{i}{2}\mathfrak{D}\right)\left(1-\frac{i}{2}\mathfrak{D}\right)^{-1},
$$
which is originally shown in \cite{Holevo_1977}.