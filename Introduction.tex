\section{Introduction}
This paper reviews results about  modular operators \cite{Longo_1978} and  Holevo's commutation operators \cite{Holevo_1977} in the simplest case.
We mainly deal with the von Neumann algebra $\mathfrak{N}=M_n(\mathbb{C})$, 
which is the ensemble of bounded linear operators on 
the Hilbert space $\cal{H}=\mathbb{C}^n$ with the inner product $(x,y)=\bar{x}^Ty$. 
Let $\rho$ be a non-degenerated density operator and 
$\omega$ be the corresponding normal state given by $\omega(A)=\mbox{Tr}\rho A$.
We regard $\mathfrak{N}$ as a Hilbert space with an inner product
\begin{equation}
\label{innerP}
\langle A, B \rangle =\frac{1}{2}\omega(BA^{\ast}+A^{\ast}B),
\end{equation}
and denote it by $\mathfrak{H}$.
In the quantum theory we consider additional bilinear form on $\mathfrak{H}$ 
\begin{equation}\label{Bform}
[A,B]=i\omega(A^{\ast}B-BA^{\ast}).
\end{equation}
Using Eqs. (\ref{innerP}) and (\ref{Bform}) we obtain 
$$
\langle X, X\rangle \geq \pm \frac{i}{2}[X,X],
$$
which yields the most general form of the uncertainty relation \cite{Holevo_1977}.
we introduce an operator $\mathfrak{D}$ by 
\begin{equation}\label{Copr}
[A,X]=\langle A, \mathfrak{D}X\rangle.
\end{equation}
The operator $\mathfrak{D}$, called a commutation operator, plays an important role in the quantum estimation theory.  

We regard $\mathfrak{N}$ as a Hilbert-Schmidt space with another inner product
$$
\langle A, B \rangle_2 =\mbox{Tr}(A^{\ast}B),
$$
and denote it by $\mathfrak{H}_2$.
Let us consider a $\ast$-representation on $\mathfrak{H}_2$
\begin{equation}\label{star-rep}
\ell :\mathfrak{N}\to \mathfrak{B}(\mathfrak{H}_2),
\end{equation}
where $\mathfrak{B}(\mathfrak{H}_2)$ is the ensemble of bounded operators on $\mathfrak{H}_2$
and $\ell(A)B=AB$.
Then the state $\omega$ can be written by the inner product $\langle \cdot, \cdot\rangle_2$ as 
$$
\omega(A)=\mbox{Tr}(\rho^{1/2}A\rho^{1/2})=\langle \rho^{1/2},\ell(A)\rho^{1/2}\rangle_2 .
$$
Now we can define the modular operator $\Delta$ for the von Neumann algebra $\mathfrak{M}=\ell(\mathfrak{N})$ and its cyclic separating vector $\rho^{1/2}$. 
We see such derived modular operator is related to the commutation operator
as
$$
    \Delta=\left(1+\frac{i}{2}\mathfrak{D}\right)\left(1-\frac{i}{2}\mathfrak{D}\right)^{-1}.
$$
