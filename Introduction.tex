\section{Introduction}
In the study of noncommutative statistics, Holevo introduced the space of square-integrable operators and the associated superoperator called the commutation operators \cite{Holevo_1977}. These are very useful mathematical tools for the solution of  noncommutative statistical problems. On the other hand the modular operator is well known in the operator theory; it appears in the main results of the Tomita-Takesaki theory. Holevo pointed out that we can show the relation between the commutation operator and the modular operator by a simple computation \cite{Holevo_1977}. 
This paper reviews results about  modular operators  and  commutation operators in the simplest case, and gives a detailed computation to verify the relation between these two operators.

We mainly deal with the von Neumann algebra $\mathfrak{N}=M_n(\mathbb{C})$, 
which is the ensemble of $n\times n$ complex matrices and can be considered as the algebra $\mathfrak{B}({\cal H})$ of bounded linear operators on 
the Hilbert space ${\cal H}=\mathbb{C}^{n}$ with the inner product $(x,y)=\bar{x}^Ty$. 
Let $\rho$ be a non-degenerated density operator and 
$\omega$ be the corresponding normal state given by $\omega(A)=\mbox{Tr}\rho A, A\in \mathfrak{N}$.
We regard $\mathfrak{N}$ as a Hilbert space with the inner product
\begin{equation}
\label{innerP}
\langle A, B \rangle =\frac{1}{2}\omega(BA^{\ast}+A^{\ast}B),
\end{equation}
and denote it by $\mathfrak{H}$.
In the quantum theory we consider additional bilinear form on $\mathfrak{H}$ 
\begin{equation}\label{Bform}
[A,B]=i\omega(A^{\ast}B-BA^{\ast}).
\end{equation}
and  obtain fundamental inequalities 
$$
\langle X, X\rangle \geq \pm \frac{i}{2}[X,X],
$$
which yield  the uncertainty relation of the most general form \cite{Holevo_1977}.
We define a commutation operator $\mathfrak{D}$ so that it satisfies 
\begin{equation}\label{Copr}
[A,X]=\langle A, \mathfrak{D}X\rangle.
\end{equation}
The operator $\mathfrak{D}$, firstly introduced by Holevo \cite{Holevo_1977}, plays an important role in the non-commutative statistical theory. From Eq. (\ref{Bform}) it holds that 
\begin{equation}
1\pm \frac{i}{2}\mathfrak{D}\geq 0.
\end{equation}
On the other hand we can also regard $\mathfrak{N}$ as a Hilbert-Schmidt space with another inner product
$$
\langle A, B \rangle_2 =\mbox{Tr}(A^{\ast}B),
$$
by virtue of its finite-dimensionality and denote it by $\mathfrak{H}_2$.
Let us consider a $\ast$-representation on $\mathfrak{H}_2$
\begin{equation}\label{star-rep}
\ell :\mathfrak{N}\to \mathfrak{B}(\mathfrak{H}_2),
\end{equation}
where $\mathfrak{B}(\mathfrak{H}_2)$ is the ensemble of bounded operators on $\mathfrak{H}_2$
and $\ell(A)B=AB$.
Then the state $\omega$ can be written by the inner product $\langle \cdot, \cdot\rangle_2$ as 
$$
\omega(A)=\mbox{Tr}(\rho^{1/2}A\rho^{1/2})=\langle \rho^{1/2},\ell(A)\rho^{1/2}\rangle_2 .
$$
%Note that the two norms induced from the inner products $\langle \cdot,\cdot \rangle$ and 
%$\langle \cdot,\cdot \rangle_2$ are related as
%\begin{equation}
%\begin{split}
%\parallel A\parallel^2 &= \langle A,A\rangle \\
%&=\frac{1}{2}(\langle \rho^{1/2}A,\rho^{1/2}A\rangle_2+\langle %A\rho^{1/2},A\rho^{1/2}\rangle_2  )\\
%&\leq \frac{1}{2}(\parallel \rho^{1/2}A\parallel_2^2+\parallel A\rho^{1/2}\parallel_2^2)\\
%&=\parallel \rho^{1/2}\parallel_{\cal H} ^2 \parallel A\parallel_2^2,
%\end{split}
%\end{equation}
%where $\parallel \cdot \parallel_{\cal H}$ is the operator norm for $\mathfrak{B}({\cal H})$
%and $\parallel \cdot \parallel_2$ is the norm for the Hilbert-Schdmit space %$\mathfrak{H}_2$. 
{\bf Remark:} In the present case we have $\mathfrak{H}_2=\mathfrak{H}$, but when we consider an infinite
dimensional Hilbert space ${\cal H}$ the equality does not hold, i.e. $\mathfrak{H}_2\subset \mathfrak{B}({\cal H})\subset \mathfrak{H}$. This may make it difficult to extend the discussion in Sec. 4 to the infinite dimensional case.\\

As stated in Sec. 3, we can define the
modular operator $\Delta$ for the von Neumann algebra $\mathfrak{M}=\ell(\mathfrak{N})$ and its cyclic separating vector $\rho^{1/2} \in \mathfrak{H}_2$. 
In this paper we derive a simple relation between such derived modular operator and the commutation operator:
$$
    \Delta=\left(1+\frac{i}{2}\mathfrak{D}\right)\left(1-\frac{i}{2}\mathfrak{D}\right)^{-1},
$$
which is originally shown in \cite{Holevo_1977}.
