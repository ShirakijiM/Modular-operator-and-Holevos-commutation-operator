\section{Modular Operator}
We give a proof  of the main results of Tomita-Takesaki theory in the case of 
$\mathfrak{M}=\ell(\mathfrak{N}) \subset \mathfrak{B}(\mathfrak{H}_2)$ with $\mathfrak{N}=M_n(\mathbb{C})$,
where all difficulties in the theory vanish.
From Eq. (\ref{lr}),
we have
\begin{equation}
\begin{split}
\mathfrak{M}^\prime&=v^\ast \tilde{\ell}(\mathfrak{N})^\prime v= v^\ast \tilde{r}(\mathfrak{N})v=r(\mathfrak{N}) \\
\mathfrak{M}^{\prime\prime}&=v^\ast \tilde{r}(\mathfrak{N})^\prime v= v^\ast \tilde{\ell}(\mathfrak{N})v=\ell(\mathfrak{N})=\mathfrak{M}.
\end{split}
\end{equation}
We introduce  a cyclic separating vector $\rho^{1/2}$, satisfying 
$$
\mathfrak{H}_2=\mathfrak{M}\rho^{1/2}=\mathfrak{M}^\prime \rho^{1/2},
$$
and consider  anti-linear operators on $\mathfrak{H}_2$ 
\begin{equation}\label{OprS}
		 S:\ell(A) \rho^{1/2} \to \ell(A)^\ast\rho^{1/2},
\end{equation}
\begin{equation}\label{OprF}
		 F:r(A) \rho^{1/2} \to r(A)^\ast\rho^{1/2}.
\end{equation}
Here 
	$$
	\ell(A)^\ast=v^\ast(A\otimes I_n)^\ast v=v^\ast \tilde{\ell}(A^\ast) v=\ell(A^\ast),
	$$
	and
	$$
  r(A)^\ast=v^\ast (I_n\otimes A^T)^\ast v=v^\ast \tilde{r}(A^\ast)v=r(A^\ast).
	$$
	The linear operator $\Delta=FS$ on $\mathfrak{H}_2$ is known as a modular operator.
For the operator $S$, we have 
	$$
    S(X)=\rho^{-1/2}X^*\rho^{1/2}.
 	$$
	In fact, putting $X=\ell(A)\rho^{1/2}=A\rho^{1/2}$,$Y=\ell(A)^\ast\rho^{1/2}=\ell(A^\ast)\rho^{1/2}=A^\ast \rho^{1/2}$,
	$$
    Y=(X\rho^{-1/2})^\ast \rho^{1/2}=\rho^{-1/2}X^\ast\rho^{1/2}.
  $$
On the other hand, for the operator $F$, we have
$$
F(X)=\rho^{1/2}X^\ast \rho^{-1/2}.
$$
In fact, putting $X=r(A)\rho^{1/2}=\rho^{1/2}A$,$Y=r(A)^\ast\rho^{1/2}=r(A^\ast)\rho^{1/2}= \rho^{1/2}A^\ast$,
	$$
    Y= \rho^{1/2}(\rho^{-1/2}X)^\ast=\rho^{1/2}X^\ast\rho^{-1/2}.
  $$
  Thus
$$
\Delta(X)=FS(X)=F(\rho^{-1/2}X^*\rho^{1/2})=\rho^{1/2}(\rho^{-1/2}X^*\rho^{1/2})^\ast \rho^{-1/2}=\rho X \rho^{-1},
$$
that is, 
\begin{equation}\label{Delta}
\Delta=v^\ast(\rho \otimes ({\rho}^{-1})^T) v.
\end{equation}
It follows that $\Delta^\ast=v^\ast (\rho\otimes({\rho}^{-1})^T)^\ast v=\Delta$.
Since $\Delta^{-1/2}=v^\ast(\rho^{-1/2}\otimes ({\rho}^{1/2})^T) v$, 
$$
\Delta^{-1/2}(X)=\rho^{-1/2}X\rho^{1/2}
$$
and hence 
$$
S(X)=\Delta^{-1/2}(X^\ast)=\Delta^{-1/2}J(X),
$$
where $J$ is an anti-linear operator defined by $J(X)=X^\ast$.
In a similar way we have
$$
F(X)=\Delta^{1/2}J(X).
$$
Since
\begin{equation}
\begin{split}
\Delta^{-it}\ell(A)\Delta^{it}(X)&=\Delta^{-it}(A\rho^{it}X\rho^{-it})=\rho^{-it}A\rho^{it}X\rho^{-it}\rho^{it}\\
                                 &=\rho^{-it}A\rho^{it}X=\ell (\rho^{-it}A\rho^{it})(X),
\end{split}
\end{equation}
we have 
$$
\Delta^{-it}\ell(A)\Delta^{it}=\ell(\rho^{-it}A\rho^{it}).
$$
On the other hand,
we have
$$
J\ell(A)J(X)=J(AX^\ast)=(AX^\ast)^\ast=XA^\ast=r(A^\ast)(X)
$$
Thus we obtain the main results of Tomita-Takesaki theory in our case: 
\begin{equation}\label{TT1}
\Delta^{-it}\mathfrak{M}\Delta^{it}=\mathfrak{M},
\end{equation}
\begin{equation}\label{TT2}
          J\mathfrak{M}J=\mathfrak{M}^{\prime}.
\end{equation}
Remark that the above discussion can be easily extend to the case where 
$\mathfrak{N}=M_{m_1}(\mathbb{C})\oplus \cdots \oplus M_{m_n}(\mathbb{C})$.
The proof of Tomita-Takesaki theory for a finite dimensional von Neumann algebra is given in the Appendix.