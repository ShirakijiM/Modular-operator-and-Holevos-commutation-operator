\section{Modular Operator}
We give a proof  of the main results of Tomita-Takesaki theory in the case of 
$\mathfrak{M}=\ell(\mathfrak{N}) \subset \mathfrak{B}(\mathfrak{H}_2)$ with $\mathfrak{N}=M_n(\mathbb{C})$,
where all difficulties in the theory vanish.
From Eq. (\ref{lr}),
we have
\begin{equation}
\begin{split}
\mathfrak{M}^\prime&=v^\ast \tilde{\ell}(\mathfrak{N})^\prime v= v^\ast \tilde{r}(\mathfrak{N})v=r(\mathfrak{N}) \\
\mathfrak{M}^{\prime\prime}&=v^\ast \tilde{r}(\mathfrak{N})^\prime v= v^\ast \tilde{\ell}(\mathfrak{N})v=\ell(\mathfrak{N})=\mathfrak{M}.
\end{split}
\end{equation}
We introduce  a cyclic and separating vector $\rho^{1/2}$, satisfying 
$$
\mathfrak{H}_2=\mathfrak{M}\rho^{1/2}=\mathfrak{M}^\prime \rho^{1/2},
$$
and consider  anti-linear operators on $\mathfrak{H}_2$ 
\begin{equation}\label{OprS}
		 S:\ell(A) \rho^{1/2} \to \ell(A)^\ast\rho^{1/2},
\end{equation}
\begin{equation}\label{OprF}
		 F:r(A) \rho^{1/2} \to r(A)^\ast\rho^{1/2}.
\end{equation}
