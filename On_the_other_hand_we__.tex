On the other hand we can also regard $\mathfrak{N}$ as a Hilbert-Schmidt space with another inner product
$$
\langle A, B \rangle_2 =\mbox{Tr}(A^{\ast}B),
$$
by virtue of its finite-dimensionality and denote it by $\mathfrak{H}_2$.
Let us consider a $\ast$-representation on $\mathfrak{H}_2$
\begin{equation}\label{star-rep}
\ell :\mathfrak{N}\to \mathfrak{B}(\mathfrak{H}_2),
\end{equation}
where $\mathfrak{B}(\mathfrak{H}_2)$ is the ensemble of bounded operators on $\mathfrak{H}_2$
and $\ell(A)B=AB$.
Then the state $\omega$ can be written by the inner product $\langle \cdot, \cdot\rangle_2$ as 
$$
\omega(A)=\mbox{Tr}(\rho^{1/2}A\rho^{1/2})=\langle \rho^{1/2},\ell(A)\rho^{1/2}\rangle_2 .
$$
%Note that the two norms induced from the inner products $\langle \cdot,\cdot \rangle$ and 
%$\langle \cdot,\cdot \rangle_2$ are related as
%\begin{equation}
%\begin{split}
%\parallel A\parallel^2 &= \langle A,A\rangle \\
%&=\frac{1}{2}(\langle \rho^{1/2}A,\rho^{1/2}A\rangle_2+\langle %A\rho^{1/2},A\rho^{1/2}\rangle_2  )\\
%&\leq \frac{1}{2}(\parallel \rho^{1/2}A\parallel_2^2+\parallel A\rho^{1/2}\parallel_2^2)\\
%&=\parallel \rho^{1/2}\parallel_{\cal H} ^2 \parallel A\parallel_2^2,
%\end{split}
%\end{equation}
%where $\parallel \cdot \parallel_{\cal H}$ is the operator norm for $\mathfrak{B}({\cal H})$
%and $\parallel \cdot \parallel_2$ is the norm for the Hilbert-Schdmit space %$\mathfrak{H}_2$. 
{\bf Remark:} In the present case we have $\mathfrak{H}_2=\mathfrak{H}$, but when we consider an infinite
dimensional Hilbert space ${\cal H}$ the equality does not hold, i.e. $\mathfrak{H}_2\subset \mathfrak{B}({\cal H})\subset \mathfrak{H}$. This may make it difficult to extend the discussion in Sec. 4 to the infinite dimensional case.
