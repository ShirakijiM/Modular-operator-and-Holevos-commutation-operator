\section{Appendix}
In this section we prove the main results of Tomita-Takesaki theorem for a finite dimensional
von Neumann algebras $\tilde{\mathfrak{N}}$ on a Hilbert space $\cal{K}$.
We assume there exists a cyclic separating vector $\xi\in \cal{K}$;
$\cal{K}=\tilde{\mathfrak{N}}\xi=\tilde{\mathfrak{N}}^\prime\xi$.
Then we define operators $\tilde{S},\tilde{F},\tilde{\Delta}$ and $\tilde{J}$ on $\cal{K}$ as
\begin{equation}
\begin{split}
\tilde{S}&:X\xi \to X^\ast \xi \\
\tilde{F}&:Y\xi \to Y^\ast \xi \\
\tilde{\Delta}&=\tilde{F}\tilde{S},\\
\tilde{J}&=\tilde{\Delta}^{1/2}\tilde{S}.
\end{split}
\end{equation}
The Wedderburn theorem states that a finite dimensional $C^{\ast}$ algebra is  $\ast$-isomorphic to a direct sum of simple matrix algebras. That is, there exists $\ast$-isomorphic function $\varphi$ for von Neumann algebra $\tilde{\mathfrak{N}}$ such that
$$
\varphi:\tilde{\mathfrak{N}}\simeq \mathfrak{N}:=M_{m_1}(\mathbb{C})\oplus \cdots \oplus M_{m_n}(\mathbb{C}).
$$
Let us consider the state on $\tilde{\mathfrak{N}}$ as
$$
\omega_\xi(A)=(\xi,A\xi)_{\cal K},\quad A\in \tilde{\mathfrak{N}},\xi\in \cal{K},
$$
where $(\cdot,\cdot)_{\cal H}$ is an inner product of the Hilbert space ${\cal H}$.
Using this state we can define the state on $\mathfrak{N}$
$$
\omega(A)=\omega_\xi(\varphi^{-1}(A)),A\in \mathfrak{N},
$$
which is normal by virtue of finite-dimensionality.
Applying the discussion in the previous sections to the von Neumann algebra $\mathfrak{N}$ and the state
$\omega$, we get
the von Neumann algebra $\mathfrak{M}=\ell (\mathfrak{N})$ on  
$\mathfrak{H}_2$, the cyclic separating vector $\rho^{1/2}\in \mathfrak{H}_2$
and the operators on $\mathfrak{H}_2$, $S, F, J$ and $\Delta$.
In particular we have
$$
\langle \ell(A)\rho^{1/2},\ell( B)\rho^{1/2}\rangle_2=(\varphi^{-1}(A)\xi,\varphi^{-1}(B)\xi)_{\cal K},
$$
which means
$$
U:\mathfrak{H}_2\ni\ell(A)\rho^{1/2}\to \varphi^{-1}(A)\xi\in {\cal K}, \quad A\in \mathfrak{N}
$$
gives a unitary operator from $\mathfrak{H}_2$ to ${\cal K}$.
Using this unitary operator we obtain the following relations
\begin{equation}
\begin{split}
\tilde{S}&=US U^{\ast},\\
\tilde{F}&=UFU^\ast ,
\end{split}
\end{equation}
and hence $\tilde{\Delta}=U\Delta U^\ast$, $\tilde{J}=UJU^\ast$.
Moreover  
$$
\tilde{\mathfrak{N}}=U\ell(\mathfrak{N})U^\ast,
$$
since $\varphi^{-1}(X)=U\ell (X) U^\ast$ for $X\in \mathfrak{N}$.
Thus we conclude the main result of Tomita-Takesaki Theory,
\begin{equation}
\begin{split}
\tilde{\Delta}^{-it}\tilde{\mathfrak{N}}\tilde{\Delta}^{it}=\tilde{\mathfrak{N}}\\
\tilde{J}\tilde{\mathfrak{N}}\tilde{J}=\tilde{\mathfrak{N}}^\prime.
\end{split}
\end{equation}


