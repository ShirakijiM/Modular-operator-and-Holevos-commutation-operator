\section{Appendix}
In this section we prove the Tomita-Takesaki theorem for a finite dimensional
von Neumann algebras $\tilde{\mathfrak{N}}$ on a Hilbert space $\cal{K}$.
We assume there exists a cyclic separating vector $\xi\in \cal{K}$;
$\cal{K}=\tilde{\mathfrak{N}}\xi=\tilde{\mathfrak{N}}^\prime\xi$.
Then we define operators $\tilde{S},\tilde{F},\tilde{\Delta}$ on $\cal{K}$ as
\begin{equation}
\begin{split}
\tilde{S}&:\tilde{\mathfrak{N}}\xi\ni X\xi \to X^\ast \xi \in \tilde{\mathfrak{N}}\xi,\\
\tilde{F}&:\tilde{\mathfrak{N}}^\prime\xi\ni Y\xi \to Y^\ast \xi \in \tilde{\mathfrak{N}}^\prime\xi,\\
\tilde{\Delta}&=\tilde{F}\tilde{S}.
\end{split}
\end{equation}
The Wedderburn theorem states that a finite dimensional $C^{\ast}$ algebra is  $\ast$-isomorphic to a direct sum of simple matrix algebras. That is, there exists $\ast$-isomorphic function $\varphi$ for von Neumann algebra $\mathfrak{N}$ such that
$$
\varphi:\tilde{\mathfrak{N}}\simeq \mathfrak{N}:=M_{m_1}(\mathbb{C})\oplus \cdots \oplus M_{m_n}(\mathbb{C})
$$

Let us consider the state on $\tilde{\mathfrak{N}}$ as
$$
\omega_\xi(A)=(\xi,A\xi)_{\mathfrak{K}},\quad A\in \tilde{\mathfrak{N}},\xi\in \cal{K}.
$$
Using this state we can define the state on $\mathfrak{N}$
$$
\omega(A)=\omega_\xi(\varphi^{-1}(A)),A\in \mathfrak{N}.
$$
Applying the discussion in the previous sections to the von Neumann algebra $\mathfrak{N}$ and 
$\omega$, we get
the von Neumann algebra $\mathfrak{M}=\ell (\mathfrak{N})$ on the Hilbert space 
$\mathfrak{H}_2$, the cyclic separating vector $\rho^{1/2}\in \mathfrak{H}_2$
and operators $S, F$ and $\Delta$ on $\mathfrak{H}_2$.

Then we obtain
$$
\langle \pi(A)\rho^{1/2},\pi( B)\rho^{1/2}\rangle_2=(\varphi^{-1}(A)\xi,\varphi^{-1}(B)\xi)_\mathfrak{K},
$$
which means
$$
U:\mathfrak{H}_2\ni\pi(A)\rho^{1/2}\to \varphi^{-1}(A)\xi\in\mathfrak{K}, \quad A\in \mathfrak{N}
$$
gives a unitary operator from $\mathfrak{H}_2$ to $\mathfrak{K}$.
This unitary operator combines $\Delta$ and $\tilde{\Delta}$ as
$$
\tilde{\Delta}=U\Delta U^{\ast},
$$
and gives the main result of Tomita-Takesaki Theory,
$$
\tilde{\Delta}^{-it}\tilde{\mathfrak{N}}\tilde{\Delta}^{it}=\tilde{\mathfrak{N}}.
$$


