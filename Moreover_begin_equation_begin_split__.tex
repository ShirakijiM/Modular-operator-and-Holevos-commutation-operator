 Moreover
	\begin{equation} \label{1+D} 
    1+\frac{i}{2}\mathfrak{D}=v^\ast [ 2(\rho\otimes I_n+I_n\otimes \rho)^{-1}  \rho \otimes I_n]v,
    \end{equation}
    \begin{equation}\label{1-D}
    1-\frac{i}{2}\mathfrak{D}=v^\ast [ 2(\rho\otimes I_n+I_n\otimes \rho)^{-1}I_n \otimes \rho^T]v,
    \end{equation}
    \begin{equation}
    1+\frac{1}{4}\mathfrak{D}^2=v^\ast[ 4(\rho\otimes I_n+I_n\otimes \rho)^{-2}\rho \otimes \rho^T]v.
	\end{equation}
From our assumption stated in Sec. 1,
$\rho^{1/2}$ is a non-degenerated operator and hence we can use it as a 
cyclic and separating vector in Sec. 3.
Thus from Eqs. (\ref{1+D}), (\ref{1-D}) and (\ref{Delta})  we conclude
		$$
    \Delta=\left(1+\frac{i}{2}\mathfrak{D}\right)\left(1-\frac{i}{2}\mathfrak{D}\right)^{-1},
		$$
		and
		$$
     \frac{i}{2}\mathfrak{D}=(\Delta-1)(\Delta+1)^{-1}.
		$$


