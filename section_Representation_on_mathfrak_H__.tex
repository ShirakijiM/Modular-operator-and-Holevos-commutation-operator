\section{Representation on $\mathfrak{H}\otimes\mathfrak{H}$}

 We can identify the Hilbert-Schmidt space $\mathfrak{H}_2=\mathfrak{N}=M_n(\mathbb{C})$ on $\mathfrak{H}=\mathbb{C}^n$ with $\mathfrak{H}\otimes\mathfrak{H}$
by a unitary operator satisfying 
$$
v(e_je_k^T)=e_j\otimes e_k
$$
where $e_j=(\delta_{j1},\delta_{j,2},...,\delta_{j,n})^T$ with the Kronecker delta $\delta_{j,l}$. 
For $\psi=\sum_j \lambda_j e_j$ and $\phi=\sum_k \mu_k e_k$
we have
$$
\begin{split}
v(\psi\phi^\ast)&=\sum_{j,k}\lambda_j\bar{\mu}_k v(e_je_k^T)\\
&=\sum_{j,k}\lambda_j\bar{\mu}_k e_j\otimes e_k={\psi}\otimes \overline{\phi}.
\end{split}
$$

For $A\in \mathfrak{N}$, $\ell(A)$ is represented as $\tilde{\ell}(A)=A\otimes I_n$ on $\mathbb{C}^n\otimes \mathbb{C}^n$.
In fact
 $$
 \begin{split}
 v(\ell(A)e_je_k^T)&=v(Ae_je_k^T)=Ae_j\otimes e_k\\
 &= (A\otimes I_n)e_k\otimes  e_j=\tilde{\ell}(A)v(e_je_k^T),
 \end{split} 
 $$
 and hence $\ell(A)=v^{\ast}\tilde{\ell}(A)v$. 
On the other hand, $r(A):\mathfrak{N}\in X \to XA\ni\mathfrak{N}$ is 
represented as $\tilde{r}(A)=I_n\otimes A^T$ on $\mathbb{C}^n\otimes \mathbb{C}^n$.
In fact
 $$
 \begin{split}
 v(r(A)e_je_k^T)&=v(e_je_k^TA)=v(e_j(A^\ast e_k)^\ast)=e_j\otimes \overline{A^\ast e_k}\\
 &= e_j \otimes A^T e_k= (I_n\otimes A^T)e_j\otimes  e_k=\tilde{r}(A)v(e_je_k^T),
 \end{split} 
 $$
 and hence $r(A)=v^{\ast}\tilde{r}(A) v$. 
