\section{Relation between Holevo's commutation operator and modular operator}

Let us see how the commutation operator defined by  (\ref{Copr}) is described on $\cal{H}\otimes \cal{H}$.
Since it holds for $A,X,Y=\mathfrak{D}X\in \mathfrak{H}(=\mathfrak{N}=\mathfrak{H}_2)$
that 
 \begin{equation}
 \begin{split}
 [A,X]&=i\omega(A^\ast X-XA^\ast)=i\mbox{Tr}\rho(A^\ast X-XA^\ast)\\
      &=\mbox{Tr}A^\ast i(X\rho-\rho X)=\langle A, i(X\rho-\rho X)\rangle_2\\
 \langle A,Y\rangle&=\omega((YA^\ast+A^\ast Y)/2)=\mbox{Tr}\rho((YA^\ast+A^\ast Y)/2)\\
 &=\mbox{Tr}A^\ast(\rho Y+Y\rho)/2=\langle A, \rho Y+Y\rho \rangle_2,
 \end{split}
 \end{equation}
  we have
 $$
 (\rho Y+Y\rho)/2=i(X\rho-\rho X),
 $$
 which can be represented on $\cal{H}\otimes \cal{H}$ as
	$$
	(\rho\otimes I_n+I_n\otimes \rho^T )v(Y)=2i(I_n\otimes \rho^T-\rho\otimes I_n)v(X).
	$$
	Thus 
	$$
	v(\mathfrak{D}X)=v(Y)=2i(\rho\otimes I_n+I_n\otimes \rho^T )^{-1}(I_n\otimes \rho^T -\rho\otimes I_n )v(X),
	$$
	that is,
	$$
	\mathfrak{D}=v^\ast[ 2i( \rho\otimes I_n+I_n\otimes \rho^T )^{-1}(I_n\otimes \rho^T -\rho\otimes I_n )]v.
	$$
	 Moreover
	\begin{equation} \label{plusD} 
    1+\frac{i}{2}\mathfrak{D}=v^\ast [ 2(\rho\otimes I_n+I_n\otimes \rho)^{-1}  \rho \otimes I_n]v,
    \end{equation}
    \begin{equation}\label{minusD}
    1-\frac{i}{2}\mathfrak{D}=v^\ast [ 2(\rho\otimes I_n+I_n\otimes \rho)^{-1}I_n \otimes \rho^T]v,
    \end{equation}
    \begin{equation}
    1+\frac{1}{4}\mathfrak{D}^2=v^\ast[ 4(\rho\otimes I_n+I_n\otimes \rho)^{-2}\rho \otimes \rho^T]v.
	\end{equation}
From our assumption stated in Sec. 1,
$\rho^{1/2}$ is a non-degenerated operator and hence we can use it as a 
cyclic and separating vector in Sec. 3.
Thus, from Eqs. (\ref{plusD}), (\ref{minusD}) and (\ref{Delta})  we conclude
		$$
    \Delta=\left(1+\frac{i}{2}\mathfrak{D}\right)\left(1-\frac{i}{2}\mathfrak{D}\right)^{-1},
		$$
		and
		$$
     \frac{i}{2}\mathfrak{D}=(\Delta-1)(\Delta+1)^{-1}.
		$$