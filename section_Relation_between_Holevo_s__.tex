\section{Relation between Holevo's commutation operator and modular operator}

Let us see how the modular operator defined by  (\ref{Copr}) is described on $\mathfrak{H}\otimes \mathfrak{H}$.
Since it holds for $A,X,Y=\mathfrak{D}X\in \mathfrak{H}(=\mathfrak{N}=\mathfrak{K})$
that 
 \begin{equation}
 \begin{split}
 [A,X]&=i\omega(A^\ast X-XA^\ast)=i\mbox{Tr}\rho(A^\ast X-XA^\ast)\\
      &=\mbox{Tr}A^\ast i(X\rho-\rho X)=\langle A, i(X\rho-\rho X)\rangle_2\\
 \langle A,Y\rangle&=\omega((YA^\ast+A^\ast Y)/2)=\mbox{Tr}\rho((YA^\ast+A^\ast Y)/2)\\
 &=\mbox{Tr}A^\ast(\rho Y+Y\rho)/2=\langle A, \rho Y+Y\rho \rangle_2,
 \end{split}
 \end{equation}
  we have
 $$
 (\rho Y+Y\rho)/2=i(X\rho-\rho X),
 $$
 which can be represented on $\mathfrak{H}\otimes \mathfrak{H}$ as
	$$
	(I_n\otimes \rho + \rho\otimes I_n)v(Y)=2i(\rho\otimes I_n -I_n\otimes \rho )v(X).
	$$
	Thus 
	$$
	v(\mathfrak{D}X)=v(Y)=2i(I_n\otimes \rho + \rho\otimes I_n)^{-1}(\rho\otimes I_n -I_n\otimes \rho )v(X),
	$$
	that is,
	$$
	\mathfrak{D}=v^\ast[ 2i(I_n\otimes \rho + \rho\otimes I_n)^{-1}(\rho\otimes I_n -I_n\otimes \rho )]v
	$$
	  Moreover
	\begin{equation}
		\begin{split}
    1+\frac{i}{2}\mathfrak{D}&=v^\ast [ 2(I_n\otimes \rho+\rho\otimes I_n)^{-1}  I_n\otimes \rho]v,\\
    1-\frac{i}{2}\mathfrak{D}&=v^\ast [ 2(I_n\otimes \rho+\rho\otimes I_n)^{-1}\rho \otimes I_n]v,\\
    1+\frac{1}{4}\mathfrak{D}^2&=v^\ast[ 4(I_n\otimes \rho+\rho\otimes I_n)^{-2}\rho \otimes \rho]v.\\

    \end{split}
	\end{equation}

	Finally	we get
		$$
    \Delta=\left(1+\frac{i}{2}\mathfrak{D}\right)\left(1-\frac{i}{2}\mathfrak{D}\right)^{-1},
		$$
		and
		$$
     \frac{i}{2}\mathfrak{D}=(1-\Delta)(1+\Delta)^{-1}
		$$

