\section{Representation on $\cal{H}\otimes\cal{H}$}

 We identify the Hilbert-Schmidt space $\mathfrak{H}_2(=\mathfrak{N}=M_n(\mathbb{C}))$ on ${\cal H}=\mathbb{C}^n$ with $\cal{H}\otimes\cal{H}$
by a unitary operator satisfying 
$$
v(e_je_k^T)=e_j\otimes e_k,
$$
where $e_j=(\delta_{j1},\delta_{j,2},...,\delta_{j,n})^T$ with the Kronecker delta $\delta_{j,l}$. 
Here, for $\psi=\sum_j \lambda_j e_j$ and $\phi=\sum_k \mu_k e_k$
we have
\begin{equation}
\begin{split}
v(\psi\phi^\ast)&=\sum_{j,k}\lambda_j\bar{\mu}_k v(e_je_k^T)\\
&=\sum_{j,k}\lambda_j\bar{\mu}_k e_j\otimes e_k={\psi}\otimes \overline{\phi}.
\end{split}
\end{equation}
In the $\ast$-representation (\ref{star-rep}), $\ell(A)$ is given by $\tilde{\ell}(A)=A\otimes I_n$ on $\cal{H}\otimes\cal{H}$.
In fact
 \begin{equation}
 \begin{split}
 v(\ell(A)e_je_k^T)&=v(Ae_je_k^T)=Ae_j\otimes e_k\\
 &= (A\otimes I_n)e_k\otimes  e_j=\tilde{\ell}(A)v(e_je_k^T),
 \end{split} 
 \end{equation}
 and hence $\ell(A)=v^{\ast}\tilde{\ell}(A)v$. 
On the other hand, 
$$
r(A):\mathfrak{N}\in X \to XA\ni\mathfrak{N}
$$ 
is represented by $\tilde{r}(A)=I_n\otimes A^T$ on $\cal{H}\otimes\cal{H}$.
In fact
 \begin{equation}
 \begin{split}
 v(r(A)e_je_k^T)&=v(e_je_k^TA)=v(e_j(A^\ast e_k)^\ast)=e_j\otimes \overline{A^\ast e_k}\\
 &= e_j \otimes A^T e_k= (I_n\otimes A^T)e_j\otimes  e_k=\tilde{r}(A)v(e_je_k^T),
 \end{split} 
 \end{equation}
 and hence $r(A)=v^{\ast}\tilde{r}(A) v$. 

Remark that $\tilde{\ell}(\mathfrak{N})=\mathfrak{N}\otimes I_n$ and $\tilde{r}(\mathfrak{N})=I_n\otimes \mathfrak{N}$ are von Neumann algebras in 
$$\mathfrak{B}({\cal H}\otimes{\cal H})=\mathfrak{B}({\cal H})\otimes \mathfrak{B}({\cal H})=\mathfrak{N}\otimes \mathfrak{N}.
$$